\documentclass[a4paper,12pt]{article}
\usepackage[utf8]{inputenc}
\usepackage{amsmath,amssymb}
\usepackage{enumitem}
\usepackage[backend=biber,style=alphabetic]{biblatex}
\addbibresource{ref.bib}

\begin{document}

\title{Proposta de Projeto}
\author{}
\date{}
\maketitle

\section*{Introdução}

O projeto é composto por duas partes: uma de computação clássica e outra de computação quântica. 
\begin{itemize}
    \item Na primeira parte, você terá como ponto de partida um código-base de uma sala de chat com criptografia autenticada simétrica. Seu objetivo será implementar um protocolo com criptografia assimétrica para a troca de chaves.
    \item Na segunda parte, seu objetivo será complementar um algoritmo de ataque à criptografia usada na troca de chaves.
\end{itemize}

\section{Parte Clássica: Troca de Chaves}

\begin{enumerate}
    \item Implemente o protocolo descrito abaixo para estabelecer uma chave compartilhada entre todos os integrantes do chat. 
    \item Explique como esse protocolo impede que alguém, ao interceptar os pacotes trocados, consiga obter a chave compartilhada.
\end{enumerate}
\subsection*{Protocolo}
Considere um conjunto de \( N \) pessoas: \( P_0, P_1, \dots, P_{N-1} \) e um servidor \( S \). Considere também uma cifra \( E, D \) de chave privada.

\begin{enumerate}
    \item O servidor $S$ escolhe um gerador \( g \) em um grupo com \( p \) elementos.

    \item Cada \( P_i \), para \( i = 0, 1, \dots, N-1 \):
    \begin{itemize}
        \item Escolhe um valor secreto \( a_i \);
        \item Calcula \( A_i = g^{a_i} \) e envia \( A_i \) para o servidor \( S \).
    \end{itemize}

    \item O servidor \( S \):
    \begin{itemize}
        \item Escolhe um valor secreto \( s \);
        \item Escolhe uma mensagem \( m \) e obtém o cifrotexto \( c = E(g^s, m) \);
        \item Para cada \( i = 0, 1, \dots, N-1 \), calcula \( A_i^{s} \) e envia este valor para \( P_i \), juntamente com \( c \).
    \end{itemize}

    \item Cada \( P_i \) (\( i = 0, 1, \dots, N-1 \)) realiza os seguintes passos:
    \begin{itemize}
        \item Recebe \( A_i^{s} \) do servidor \( S \);
        \item Calcula \( \left(A_i^{s}\right)^{a_i^{-1}} = g^{s} \). \text{Para este passo, é recomendável que \( a_i \) seja um número primo.}
        \item Calcula \( m'_i = D(g^s, c) \) e envia \( m'_i \) para \( S \).
    \end{itemize}

    \item O servidor \( S \) compara \( m'_i \) com \( m \), para cada \( i = 0, 1, \dots, N-1 \), e retorna "ok" para as pessoas que tiveram \( m'_i = m \) e "not ok" para aquelas cujo resultado foi diferente. 

    \item Cada \( P_i \), para \( i = 0, 1, \dots, N-1 \), analisa a resposta de \( S \):
    \begin{itemize}
        \item Se recebeu "ok", o protocolo é finalizado.
        \item Caso contrário, retorna ao passo 2.
    \end{itemize}
\end{enumerate}

O valor \( g^{s} \) é então a chave coletiva compartilhada entre todos os participantes.

\cite{boneh2020graduate}

\section{Parte Quântica: Ataque à Criptografia}
\begin{itemize}
    \item Implementar um algoritmo que encontre o período de uma exponenciação modular, utilizando as bibliotecas de sua preferência (Qiskit, Pennylane, etc.).
    \item Explicar como um atacante com um computador quântico (com um número suficiente de qubits confiáveis) poderia descobrir a chave compartilhada utilizando um algoritmo que determina o período de uma função.
\end{itemize}

\cite{website_example}
\printbibliography

\end{document}
